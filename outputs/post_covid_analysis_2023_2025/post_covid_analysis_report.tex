\documentclass[11pt,a4paper]{article}
\usepackage[utf8]{inputenc}
\usepackage[T1]{fontenc}
\usepackage{geometry}
\usepackage{amsmath,amssymb}
\usepackage{booktabs}
\usepackage{array}
\usepackage{longtable}
\usepackage{graphicx}
\usepackage{float}
\usepackage{xcolor}
\usepackage{hyperref}
\usepackage{bookmark}
\usepackage{fancyhdr}

\geometry{margin=1in}
\pagestyle{fancy}
\fancyhf{}
\rhead{Brazilian REH Analysis}
\lfoot{\today}
\rfoot{\thepage}

\definecolor{academicblue}{HTML}{2E86AB}
\definecolor{academicred}{HTML}{C73E1D}

\title{Brazilian Inflation Expectations Rationality: Post-COVID Recovery Analysis (2023-2025)}
\author{Brazilian REH Analyzer v2.0.0}
\date{\today}

\begin{document}
\maketitle

\section{Executive Summary}

\begin{center}
\fbox{\begin{minipage}{0.9\textwidth}
\textbf{\large Analysis Overview}\\[0.5em]
\textcolor{academicred}{\textbf{Rational Expectations Hypothesis: FAIL}}\\[0.3em]
\textbf{Analysis Period:} 2023-01-02 to 2024-06-28\\[0.2em]
\textbf{Observations:} 373\\[0.2em]
\textbf{Mean Forecast Bias:} -3.875 p.p.\\[0.2em]
\textbf{Bias Severity:} Extreme (Overestimation)\\[0.2em]
\end{minipage}}
\end{center}

\section{Comprehensive Descriptive Statistics}

\begin{table}[H]
\centering
\caption{Comprehensive Statistical Summary}
\begin{tabular}{lccc}
\toprule
\textbf{Statistic} & \textbf{Forecast (\%)} & \textbf{Realized (\%)} & \textbf{Error (p.p.)} \\
\midrule
Mean & 4.290 & 0.415 & -3.875 \\
Median & 3.963 & 0.390 & -3.655 \\
Std. Deviation & 0.760 & 0.278 & 0.812 \\
Minimum & 3.358 & -0.020 & -5.700 \\
Maximum & 5.860 & 1.310 & -2.363 \\
Skewness & 0.917 & 1.504 & -0.482 \\
Kurtosis & -0.704 & 3.214 & -0.637 \\
Observations & 373 & 373 & 373 \\
\bottomrule
\end{tabular}
\end{table}

\section{Rationality Test Results}

\begin{table}[H]
\centering
\caption{REH Test Results Summary}
\begin{tabular}{lcc}
\toprule
\textbf{Test} & \textbf{Result} & \textbf{Implication} \\
\midrule
Unbiasedness & \textcolor{academicred}{FAIL} & Systematic bias \\
Mincer-Zarnowitz & \textcolor{academicred}{FAIL} & Forecast efficiency \\
Efficiency & \textcolor{academicred}{FAIL} & Information usage \\
Overall REH & \textcolor{academicred}{FAIL} & Rational expectations \\
\bottomrule
\end{tabular}
\end{table}

\section{Mincer-Zarnowitz Regression Analysis}

The Mincer-Zarnowitz regression tests the null hypothesis of rational expectations:
\begin{equation}
P_t = \alpha + \beta \cdot E_{t-12}[P_t] + \varepsilon_t
\end{equation}
where $H_0: (\alpha, \beta) = (0, 1)$ under rational expectations.

\begin{table}[H]
\centering
\caption{Mincer-Zarnowitz Regression Results}
\begin{tabular}{lccccc}
\toprule
\textbf{Parameter} & \textbf{Estimate} & \textbf{Std. Error} & \textbf{t-stat} & \textbf{p-value} & \textbf{95\% CI} \\
\midrule
$\alpha$ (Intercept) & 0.433 & 0.000 & 5.23 & 0.0000 & [0.270, 0.596] \\
$\beta$ (Slope) & -0.004 & 0.000 & -0.22 & 0.8248 & [-0.042, 0.033] \\
\bottomrule
\end{tabular}
\end{table}

\textbf{Model Diagnostics:} $R^2 = 0.0001$, Joint F-statistic = 37464.85 (p = 0.000000)

\subsection{Economic Interpretation}
\begin{itemize}
\item $\alpha = 0.433 \neq 0$: Systematic forecast bias detected
\item $\beta = -0.004 \neq 1$: Forecasters under-respond to their predictions
\item Joint test rejection indicates violations of both unbiasedness and efficiency
\end{itemize}

\section{Structural Break Analysis}

\begin{table}[H]
\centering
\caption{Sub-period Analysis Results}
\begin{tabular}{lcccc}
\toprule
\textbf{Period} & \textbf{Start} & \textbf{End} & \textbf{Mean Error} & \textbf{REH Status} \\
\midrule
Period 1 & 2023-01-02 & 2023-06-30 & -4.827 & \textcolor{academicred}{FAIL} \\
Period 2 & 2023-07-03 & 2023-12-28 & -3.623 & \textcolor{academicred}{FAIL} \\
Period 3 & 2023-12-29 & 2024-06-28 & -3.180 & \textcolor{academicred}{FAIL} \\
\bottomrule
\end{tabular}
\end{table}

\subsection{Structural Break Interpretation}
\begin{itemize}
\item Forecast bias ranges from -4.827 to -3.180 p.p. across sub-periods
\item Total bias variation: 1.647 p.p.
\item \textbf{Substantial} time-variation in forecast bias detected
\end{itemize}

\section{Economic Interpretation}

\subsection{Quantitative Bias Assessment}
\begin{table}[H]
    \centering
    \caption{Enhanced Bias Analysis}
    \begin{tabular}{lcc}
        \toprule
        \textbf{Metric}      & \textbf{Value} & \textbf{Assessment} \\
        \midrule
        Direction            & Overestimation & --                  \\
        Magnitude            & 3.875 p.p.     & Extreme             \\
        Grade Category       & F              & High Impact         \\
        Bias Ratio           & 4.77           & High Dominance      \\
        Systematic Component & 97.9\%         & of Total Error      \\
        \bottomrule
    \end{tabular}
\end{table}

\subsection{Quantitative Efficiency Assessment}
\begin{table}[H]
    \centering
    \caption{Enhanced Efficiency Analysis}
    \begin{tabular}{lcc}
        \toprule
        \textbf{Metric}        & \textbf{Value} & \textbf{Assessment} \\
        \midrule
        Ljung-Box Statistic    & 3327.5         & Low                 \\
        LB p-value             & 1.0000         & Not Significant     \\
        Efficiency Score       & 50.0/100       & Poor                \\
        Predictability Index   & 33.28          & High Predictability \\
        Information Processing & Poor           & Quality Assessment  \\
        \bottomrule
    \end{tabular}
\end{table}

\subsection{Enhanced Mincer-Zarnowitz Coefficient Analysis}
\textbf{Alpha Coefficient Interpretation:}\\
$\alpha = 0.433$ (95\% CI: [0.000, 0.000])\\
\textit{small systematic over-prediction of 0.433 percentage points}

\textbf{Beta Coefficient Interpretation:}\\
$\beta = -0.004$ (95\% CI: [0.000, 0.000])\\
\textit{forecasters systematically move opposite to reality ($\beta$ = -0.004), indicating severe misinterpretation}

\textbf{Rationality Plausibility Assessment:}\\
$\alpha = 0$ plausible: Yes\\
$\beta = 1$ plausible: No\\
Joint rationality plausible: No

\subsection{Comprehensive Assessment Dashboard}
\begin{table}[H]
    \centering
    \caption{Comprehensive Quality Assessment}
    \begin{tabular}{lcc}
        \toprule
        \textbf{Assessment Dimension} & \textbf{Value}                    & \textbf{Category} \\
        \midrule
        Overall Quality Score         & 21.9/100                          & Very Poor         \\
        Root Mean Square Error        & 3.959 p.p.                        & Accuracy Measure  \\
        Mean Absolute Error           & 3.875 p.p.                        & Precision Measure \\
        R-Squared                     & 0.000                             & 0.0\% Explained   \\
        REH Compatibility             & \textcolor{academicred}{REJECTED} & Weak Evidence     \\
        \bottomrule
    \end{tabular}
\end{table}

\subsection{Policy Scenario Analysis}
Following 2024 central bank forecasting standards (Bernanke Review), we present scenario-based assessments:

\textbf{Current Persistence} (Probability: 70\%):\\
\textit{Bias and inefficiencies persist at current levels}\\
Expected MAE: 4.07 p.p., Priority: Immediate Intervention Required

\textbf{Gradual Improvement} (Probability: 20\%):\\
\textit{Forecasting quality improves over 2-3 years}\\
Expected MAE: 2.71 p.p., Priority: Supportive Measures

\textbf{Deterioration} (Probability: 10\%):\\
\textit{Forecasting quality deteriorates further}\\
Expected MAE: 5.04 p.p., Priority: Crisis Intervention

\subsection{Key Quantitative Insights}
\begin{itemize}
    \item Bias magnitude: 3.87 percentage points
    \item Efficiency loss: 100.0\% of variation unexplained
    \item Predictable error component: 97.1\% of total error
\end{itemize}

\section{Enhanced Policy Implications}

Following 2024 forecast evaluation standards with quantitative evidence-based recommendations.

\subsection{For Central Bank Policymakers}
\textbf{Quantitative Evidence-Based Recommendations:}

\begin{itemize}
\item QUANTIFIED BIAS: Systematic overestimation of 3.87 percentage points requires immediate attention
\item EFFICIENCY TARGET: Current autocorrelation statistic of 3328 needs reduction to <20 for acceptable efficiency
\item QUALITY SCORE: Current forecast quality score of 21.9/100 indicates urgent intervention required
\item CRITICAL: Negative $\beta$ coefficient (-0.004) indicates forecasters systematically misinterpret central bank signals
\item Address systematic bias of 3.87 p.p. through enhanced communication
\item Target efficiency improvements to reduce autocorrelation from 3328
\item Implement forecaster training programs
\end{itemize}

\textbf{Specific Performance Targets:}
\begin{itemize}
\item Reduce systematic bias from 3.87 to <2.71 percentage points within 24 months
\item Improve efficiency from current LB statistic of 3328 to <20 within 18 months
\end{itemize}

\subsection{For Market Participants}
\textbf{Quantified Market Opportunities:}

\begin{itemize}
\item ARBITRAGE OPPORTUNITY: Predictable bias of 3.87 p.p. offers systematic profit potential
\item ERROR PREDICTABILITY: 97.1% of forecast errors are predictable, violating market efficiency
\item RISK ASSESSMENT: Quality score of 21.9/100 suggests high uncertainty in market-based expectations
\end{itemize}

\textbf{Risk-Return Assessment:}
\begin{itemize}
\item Strategy Risk Level: High (Quality Score: 21.9/100)
\item Expected Volatility: 3.96 percentage points RMSE
\item \textcolor{academicred}{WARNING: Very poor forecast quality increases strategy risk}
\end{itemize}

\subsection{For Researchers}
\textbf{Research Priorities with Statistical Evidence:}

\begin{itemize}
    \item PERSISTENCE: REH violations documented over 1.5-year period with consistent patterns
    \item MODEL SPECIFICATION: R² of 3.959 suggests -295.9% of variation unexplained
    \item ALTERNATIVE MODELS: Evidence strongly supports adaptive expectations framework
\end{itemize}

\textbf{Model Development Priorities:}
\begin{itemize}
    \item \textbf{URGENT:} Investigate counter-intuitive negative $\beta$ coefficient - suggests fundamental model misspecification
    \item Low explanatory power (R² = 0.000) suggests need for alternative theoretical frameworks
\end{itemize}

\subsection{Scenario-Based Implementation Strategy}
\textbf{Recommended approach based on probabilistic scenarios:}

\begin{enumerate}
\item \textbf{Current Persistence} (70\% probability):
Priority Level: Immediate Intervention Required
\begin{itemize}
    \item Address systematic bias of 3.87 p.p. through enhanced communication
    \item Target efficiency improvements to reduce autocorrelation from 3328
    \item Implement forecaster training programs
\end{itemize}
\item \textbf{Gradual Improvement} (20\% probability):
Priority Level: Supportive Measures
\begin{itemize}
    \item Monitor improvement trends and adjust communication strategy
    \item Phase in advanced forecasting methodologies
    \item Maintain current policy support
\end{itemize}
\item \textbf{Deterioration} (10\% probability):
Priority Level: Crisis Intervention
\begin{itemize}
\item Emergency review of forecasting infrastructure
\item Consider alternative expectation anchoring mechanisms
\item Implement mandatory forecaster recalibration
\end{itemize}
\end{enumerate}

\subsection{Recommended Implementation Timeline}
\textbf{Evidence-based priority sequence:}

\begin{description}
    \item[Immediate (0-6 months):] Address most severe biases and communication failures
    \item[Short-term (6-18 months):] Implement efficiency improvements and forecaster training
    \item[Medium-term (18-36 months):] Monitor improvements and adjust strategies based on scenario outcomes
    \item[Long-term (36+ months):] Evaluate fundamental model changes if improvements insufficient
\end{description}

\end{document}
